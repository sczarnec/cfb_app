% Options for packages loaded elsewhere
\PassOptionsToPackage{unicode}{hyperref}
\PassOptionsToPackage{hyphens}{url}
%
\documentclass[
]{article}
\usepackage{amsmath,amssymb}
\usepackage{iftex}
\ifPDFTeX
  \usepackage[T1]{fontenc}
  \usepackage[utf8]{inputenc}
  \usepackage{textcomp} % provide euro and other symbols
\else % if luatex or xetex
  \usepackage{unicode-math} % this also loads fontspec
  \defaultfontfeatures{Scale=MatchLowercase}
  \defaultfontfeatures[\rmfamily]{Ligatures=TeX,Scale=1}
\fi
\usepackage{lmodern}
\ifPDFTeX\else
  % xetex/luatex font selection
\fi
% Use upquote if available, for straight quotes in verbatim environments
\IfFileExists{upquote.sty}{\usepackage{upquote}}{}
\IfFileExists{microtype.sty}{% use microtype if available
  \usepackage[]{microtype}
  \UseMicrotypeSet[protrusion]{basicmath} % disable protrusion for tt fonts
}{}
\makeatletter
\@ifundefined{KOMAClassName}{% if non-KOMA class
  \IfFileExists{parskip.sty}{%
    \usepackage{parskip}
  }{% else
    \setlength{\parindent}{0pt}
    \setlength{\parskip}{6pt plus 2pt minus 1pt}}
}{% if KOMA class
  \KOMAoptions{parskip=half}}
\makeatother
\usepackage{xcolor}
\usepackage[margin=1in]{geometry}
\usepackage{graphicx}
\makeatletter
\def\maxwidth{\ifdim\Gin@nat@width>\linewidth\linewidth\else\Gin@nat@width\fi}
\def\maxheight{\ifdim\Gin@nat@height>\textheight\textheight\else\Gin@nat@height\fi}
\makeatother
% Scale images if necessary, so that they will not overflow the page
% margins by default, and it is still possible to overwrite the defaults
% using explicit options in \includegraphics[width, height, ...]{}
\setkeys{Gin}{width=\maxwidth,height=\maxheight,keepaspectratio}
% Set default figure placement to htbp
\makeatletter
\def\fps@figure{htbp}
\makeatother
\setlength{\emergencystretch}{3em} % prevent overfull lines
\providecommand{\tightlist}{%
  \setlength{\itemsep}{0pt}\setlength{\parskip}{0pt}}
\setcounter{secnumdepth}{-\maxdimen} % remove section numbering
\ifLuaTeX
  \usepackage{selnolig}  % disable illegal ligatures
\fi
\usepackage{bookmark}
\IfFileExists{xurl.sty}{\usepackage{xurl}}{} % add URL line breaks if available
\urlstyle{same}
\hypersetup{
  pdftitle={CFB Project Summary},
  pdfauthor={Steve Czarnecki},
  hidelinks,
  pdfcreator={LaTeX via pandoc}}

\title{CFB Project Summary}
\author{Steve Czarnecki}
\date{2024-12-20}

\begin{document}
\maketitle

\subsection{The Model}\label{the-model}

I wanted to build a point differential model for CFB Games. The
intention was to 1) predict games for fun and 2) try to win money vs
sports books on spread and/or moneyline bets. The second goal is hard
but this is my attempt.

I used cfbfastr to aggregate a bunch of data from previous games, using
rolling means from both play-by-play and game-by-game data. In the end,
I threw 109 predictors into an XGBoosted model and tuned.

The v1 results were fine. RMSE was at 17.4, which is deceivingly low due
to the difficulty of predicting game results. After converting to binary
predicted wins/losses, the accuracy was 72.6\%, short of sportsbooks'
74.6\%. Finally, our accuracy when using our model to bet the spread was
52.03\%, .5\% short of the 52.5\% mark I projected was needed to break
even vs sportsbooks.

From here, I wanted to test the data more to see what our actual returns
would be, using actual historical odds and spreads. I also wanted to
make it easier to predict future games.

\subsection{The App}\label{the-app}

The goal of the app was to 1) create a dynamic dashboard-like page to
evaluate the model's betting performance and 2) create an environment
where I can easily predict future games.

Starting out, I built the betting analysis page. This allows the user to
see how the model performs when betting on either the spread or
moneyline on our test data (everything that wasn't used in training).
The return on the spread is actually a lot better than expected at
-.28\% while moneyline is not great at -7\%. Compared to the ``average
bettor'' baseline of -4.53\%, the model performs very well, almost
breaking even. The user can also adjust the filters to see how well it
performs in certain scenarios, which will be interesting to see
(e.g.~where can I improve model, where should I be betting
specifically).

Next, I built the game predictor. The user can input any combination of
two teams into the page and get an output of projected point
differential in the matchup. Here, I can easily adjust if I want to see
my model's projections for the playoff games, for example, or if I just
want to see a random score.

The final two pages will be coming out soon. The first will have an
adjustable playoff feature, where you can input any 12 teams into the
bracket and it will spit out who wins each game and eventually the
entire playoff. The next will be a current week page where all of the
games for the current week are predicted.

\subsection{Going Forward}\label{going-forward}

After I finish the app's final pages, I will be adjusting the model to
make it better. There are a lot of predictors that I did not have enough
time to include in v1. The biggest is sportsbook data, which was hard to
find. It will be in the next version so I can build off the sportsbooks'
spreads. After I get these predictors in and re-tune, I will spit out
the new model, adjust the app accordingly, and see how much better v2
performs.

In the end, I don't expect to actually be able to bet with this.
Sportsbooks are very good at setting lines, and even if I'm confident of
the external validity of my test data, my returns will likely only be 1
or 2\% at most. Since I don't have thousands of dollars to throw at CFB
games, this model won't be very practical betting-wise. It will be cool
just to have a model that outperforms sportsbook spreads and can give me
good estimates on games in the future.

\end{document}
